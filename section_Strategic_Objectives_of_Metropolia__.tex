\section{Strategic Objectives of Metropolia UAS}

Helsinki Metropolia University of Applied Sciences is currently in a major curriculum reform process. 
Instead of traditional teaching with separated courses, starting with the first year students in 2014, the teaching is now organized in course modules or projects. 
For example, the bachelor of engineering in information technologies 1st year studies is now divided into four modules\footnote{see \url{http://opinto-opas.metropolia.fi/fileadmin/user_upload/En/General/New_student_2015/Information_Technology__Full-time_studies/TXL15S1_1stYearStudies.pdf}} of 15ECTS credits. 
Some goals of the reform are to reduce the amount of student missing just few credits to graduate (e.g. missing one mandatory physics course), avoid course overlapping, to better integrate separated topics together and most importantly, to reduce the amount of students who interrupt their studies.

In this modules the students are always together all day, five day a week. A group of teacher from various disciplines such as mathematics, programming, communication instructing them. 
For example, I was part of the Networks module where the students learn and practice with the basics of data communication with four different teachers from the hardware (cables, antenna,\ldots), protocols, security and applications (web pages). 
At the same time, they learn the mathematics related to the topics like calculating antenna power and distance of emission, geometry to apply transformation in web pages,\ldots 
They also had English communication where they practiced by performing presentations about what they learned in the module.

Holvikivi et al. \cite{holvikivi_2015} present the early findings of the curriculum reform in engineering education in the Metropolia UAS. 
They demonstrate that the new curriculum shows increased retention rate and that the student feedback on the working methods as well as teaching and guidance during the courses were viewed as positive. They also points weaknesses such as unbalanced workload for the students and difficult subject matters. Holvikivi et al. point out teachers feedback which raise organizational problem and that teachers were not used to co-teaching.

Hjort et al. \cite{hjort_2015} study student self-reliance and collaborative practices in project-based learning. 
They demonstrate that collaborative work around a shared objective is an efficient and inspiring mode of study.
They point out that students quickly adopted a self-directed mode of operation, helping each other instead of asking teachers and adopted team working skills very fast, even for students who never worked in team before.

  
  
  